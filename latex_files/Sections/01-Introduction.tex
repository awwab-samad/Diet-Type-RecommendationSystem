% Introduction
\section{Introduction}
With the increasing prevalence of chronic diseases such as diabetes, cardiovascular disorders, and obesity leading to substantial morbidity and mortality rates, there is a growing need for effective strategies to manage and prevent these conditions through lifestyle interventions, especially diet modification. However, devising personalized dietary plans that cater to individuals' unique medical profiles can be challenging, requiring a comprehensive understanding of their health status and nutritional requirements. Traditional approaches to dietary recommendation often rely on general guidelines and manual assessment by healthcare professionals, which may not fully account for individual variability and evolving health conditions often hindering effective disease prevention strategies.

In recent years, machine learning techniques have emerged as promising tools for personalized healthcare, offering the potential to analyze large volumes of medical data and extract meaningful insights to inform clinical decision-making. By leveraging algorithms such as neural networks, which are capable of learning complex, nonlinear patterns from data, it becomes possible to develop predictive models that can tailor dietary recommendations to each individuals' needs.

This project seeks to explore the feasibility of using Multi-Layer Perceptrons (MLP) to generate personalized dietary recommendations based on individuals' medical data. By training an MLP model on a diverse dataset comprising various medical parameters and corresponding dietary recommendations, we aim to analyze key risk factors and biomarkers associated with chronic diseases and develop a predictive system capable of providing optimal dietary choices for individuals that could ultimately encourage individuals to make healthier dietary choices and reduce their risk of chronic diseases.